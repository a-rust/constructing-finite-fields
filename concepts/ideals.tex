\section{Ideals}
\begin{definition}
A (two-sided) ideal $A$ of a ring $R$ is a subring of $R$ that satisfies the following:

\begin{enumerate}
    \item For all $a\in A$, $r\in R$, $a*r, r*a \in A$
\end{enumerate}
\end{definition}

The second requirement is known as an "absorbing" property. Consider the figure below. Suppose the outer-green circle is a ring $R$ with an arbitrary element $r\in R - A$, and the inner-blue circle is an ideal of $R$, namely $A$, with an arbitrary element $a$. The absorbing property really means that if you multiply $r$ and $a$ (either $r*a$ or $a*r$), then the result must be an element of $A$; the ideal $A$ absorbs elements in $R - A$.


\begin{center}
    \begin{tikzpicture}[
squarednode/.style={rectangle, draw=blue!60, fill=red!5, very thick, minimum size=5mm},
]
        \draw[blue] (-10,0) circle [radius=1cm];
        \draw[green] (-10,0) circle [radius=2cm];
\node[squarednode] at (-10,0)     (maintopic)                              {a};
\node[squarednode] (uppercircle)       [above=of maintopic] {r};
\end{tikzpicture}
\end{center}

Some ideals of a ring $R$ can be "generated" by element(s) of $R$. More formally, an ideal $A$ of a ring $R$ generated by an element $a \in R$ is the set $$<a> = \{a*r : r \in R\}.$$ 
\hspace{\parindent}These will be useful when discussing irreducible polynomials.

We now introduce a special type of ideal, a maximal ideal. 
\begin{definition}
A maximal ideal $A$ of a ring $R$ is an ideal that satisfies the following:
\begin{enumerate}
    \item For any ideal $B$ of $R$ where $A \subset B$, either $A = B$ or $B = R$.
\end{enumerate}
\end{definition}

This definition is basically saying that for an ideal $A$ to be a maximal ideal of a ring $R$, $A$ cannot be contained within any proper subset of $R$. This along with the following definition are VERY important for setting up the construction of finite fields.

\begin{definition}
Let $A$ be an ideal of a ring $R$. The set $$R/A = \{r + A | r \in A\}$$ is known as a quotient ring. Its binary operations addition and multiplication are defined by:
\begin{itemize}
    \item For all $(r_m + A), (r_n + A) \in R/A$,

\begin{enumerate}
    \item $(r_m + A) + (r_n + A) = (r_m + r_n + A)$
    \item $(r_m + A)*(r_n + A) = (r_m*r_n + A)$
\end{enumerate}
\end{itemize}
\end{definition}

For now, just know that specific quotient rings are closely related to fields.





