\section{Constructing a finite field from a quotient ring}

The following theorems are all we need to finally prove that a (finite) field can be constructed from a quotient ring

\begin{theorem}
The ideal generated by an irreducible polynomial is maximal.
\end{theorem}

\begin{theorem}
Let $A$ be a maximal ideal of a ring $R$. Then $R/A$ is a field.
\end{theorem}

And there we have it. The first theorem states that if we have an irreducible element $p(x)$ in some polynomial ring $F[x]$, then $<p(x)>$ is an ideal of $F[x]$. Specifically, $<p(x)>$ is a maximal ideal. Thus, by the second theorem, we obtain the main result of these notes.

\begin{result}
When $p(x)$ is an irreducible polynomial over $F$, $F[x]/<p(x)>$ is a field. 
\end{result}

\subsection{Irreducibility tests}
While the result may seem powerful, we need to actually find irreducible polynomials to construct anything.

Luckily, there are lots of irreducibility tests, which often depend on the field of coefficients. We will only cover one that's simple, but still very useful.

\begin{theorem}
Let $p(x)$ be a polynomial over some field $F$ with the degree of $p(x)$ being 2 or 3. If $p(x)$ has no roots (zeros) in $F$, then $p(x)$ is irreducible. 
\end{theorem}

A root (or zero) of a polynomial is an element $a$ such that $p(a) = 0$. For example, $x+2 \in \mathbf{Q}[x]$ has a root of $-2$ (in $\mathbf{Q}$), since $(-2) + 2 = 0$. 

On the other hand, $x^2 - 2$ has no roots in $\mathbf{Q}$, since no element $a$ in $\mathbf{Q}$ satisfies $(a)^2 - 2 = 0$.

So we only need to find a degree 2 (or 3) polynomial $p(x)$, and make sure it doesn't have any roots; then we are certain that $p(x)$ is irreducible. But what approach can we take to check that every single possible element in the field of coefficients is not a root? With "small" fields, we can exhaustively check all options. 

Consider $p(x) = x^2 + x + 1$ over the field $\mathbf{Z}_2$. Since $\mathbf{Z}_2$ only has 3 elements ($0, 1$), we only need to make sure none of the elements are roots of $p(x)$.

\begin{itemize}
    \item $(0)^2 + (0) + 1 = 1$
    \item $(1)^2 + (1) + 1 = 1$
\end{itemize}

Note the arithmetic is done modulo 3. Since none of these equations equal 0, we know $x^2 + x + 1$ is irreducible over $\mathbf{Z}_2$. Thus, by the main result of this paper, $\mathbf{Z}_2[x]/<x^2+x+1>$ is a field.
But is this a finite field? Yes! But how do we know?

\subsection{Elements of a quotient ring}

What do elements look like in $\mathbf{Z}_2[x]/<x^2+x+1>$? 

By definition, 

$$\mathbf{Z}_2[x]/<x^2+x+1> = \{a + <x^2 + x + 1> : a \in Z_3 \}.$$

Which means the following are some elements of $\mathbf{Z}_2[x]/<x^2+x+1>$:

\begin{enumerate}
    \item $0 + <x^2 + x + 1> = <x^2 + x + 1>$
    \item $1 + <x^2 + x + 1>$
    \item $x + 0 + <x^2 + x + 1> = x + <x^2 + x + 1>$
    \item $x + 1 + <x^2 + x + 1>$
\end{enumerate}

Why did I stop at $x + 1 + <x^2 + x + 1>$? Notice that $<x^2 + x + 1>$ acts as the additive identity (0) of the field $\mathbf{Z}_2[x]/<x^2+x+1>$. Also notes that $x^2 + x + 1$ and $0$ are BOTH elements of $<x^2 + x + 1>$. Thus, $x^2 + x + 1 + <x^2 + x + 1> = 0 + <x^2 + x + 1>$ (for more information, look up cosets). It then follows that $x^2 + x + 1 = 0$, which implies that $x^2 = -x - 1 = x + 1$ (since we are in modulo 2).

Therefore, $x^2 + <x^2 + x + 1> = x + 1 + <x^2 + x +1>$, which is already a defined element above. In fact, it is the case that all polynomials of degree 2 or more will reduce to one of the above defined elements of $\mathbf{Z}_2[x]/<x^2+x+1>$. So this field is in fact finite (with exactly the $4$ elements above). Finally, we have constructed a finite field of $4$ elements. And this cannot be isomorphic to $\mathbf{Z}_4$, because $\mathbf{Z}_4$ is not even a field (consider $2 \neq 0$, which has no multiplicative inverse). This is just one of many different examples for constructing a finite field. 



