\section{Rings}

\hspace{\parindent}Every ring is an abelian group, along with some additional requirements. Some authors define a ring differently from others. For these notes, we're going to use the following definition:

\begin{definition}
A ring is a set $R$, along with two distinct binary operations, addition (+) and multiplication ($*$), that satisfies the following:
\begin{enumerate}
    \item $R$ is an abelian group under $+$ with the additive identity $0$
    \item $R$ is associative under $*$
    \item $*$ distributes over $+$ in $R$. That is, for all $a, b, c \in R$, 
        $$a*(b+c) = (a*b+a*c)$$
        $$(a*b)*c = (a*c+b*c)$$

\end{enumerate}
\end{definition}

Technically, the binary operations can be whatever you want, like with rings. But for simplicity, we will use addition and multiplication.

Note that there is no requirement for every element in a ring $R$ to have a multiplicative inverse. Those elements with multiplicative inverses are called \emph{units} of $R$.

Similar to the notation for groups, we can reference a ring by the tuple (set name, +, *). So for example, $(R, +, *)$. But since we are only considering addition and multiplication, we can just refer to the ring by its name.

\begin{center}
    \underline{Examples of rings}:
\end{center}

\begin{itemize}
    \item $Z_n$ for some $n\in\mathbf{Z}$
    \item $\mathbf{Q} = \{\frac{a}{b} : a, b\in\mathbf{Z}\}$
    \item $\mathbf{Q}[x] = \{a_0 + a_1(x) + a_2(x^2) + \dots : a_i \in \mathbf{Q}\}$
\end{itemize}

The last example, $\mathbf{Q}[x]$, is the set of polynomials with rational coefficients. Polynomial rings are VERY important for the construction of finite fields. Just keep that in mind for now.

Just like abelian groups, there exist abelian (commutative) rings. The only additional requirement is that the ring satisfies commutativity under multiplication.
    
\begin{definition}
A subring $S$ of a ring $R$ is a subset of $R$ and satisfies the following:
\begin{enumerate}
    \item $S$ is itself a ring under the same operations as $R$.
\end{enumerate}
\end{definition}

\begin{center}
    \underline{Example of a subring}:
\end{center}

\begin{itemize}
    \item $2\mathbf{Z} = \{2x: x\in\mathbf{Z}\}$ is a subring of $\mathbf{Z}$
\end{itemize}

$\mathbf{Z}_{10}$ has completely different binary operations than $\mathbf{Z}$. Specifically, the additive and multiplication operations $\mathbf{Z}_{10}$ are addition mod 10, and multiplication mod 10, respectively, which is different from normal addition and multiplication.
