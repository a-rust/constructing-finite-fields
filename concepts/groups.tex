\section{Groups}

\begin{definition}
A group is a set $G$, along with a binary operation \emoji{sparkle}, that satisfies the following conditions:

\begin{enumerate}

  \item $G$ is closed under \emoji{sparkle}️. That is, for all $a, b \in G$, 
  
  $a$\emoji{sparkle}$b \in G$.
  \item $G$ satisfies associativity under \emoji{sparkle}. That is, for all $a, b, c \in G$,
    
  ($a$\emoji{sparkle}$b$)\emoji{sparkle}$c$ = $a$\emoji{sparkle}($b$\emoji{sparkle}$c$) 
  
  \item There exists an identity element in $G$. That is, there exists an element $e \in G$ such that, for all $a \in G$, 
  
  $a$\emoji{sparkle}$e$ = $e$\emoji{sparkle}$a$ = $a$
  
  \item Every element $a\in G$ has an inverse $a' \in G$. That is,
  
  $a$\emoji{sparkle}$a'$ = $a'$\emoji{sparkle}$a$ = $e$
  
\end{enumerate}

\end{definition}

We denote a group by the tuple $(G,$ \emoji{sparkle}$)$.

\begin{center}
    \underline{Examples of groups}:
\end{center}

\begin{itemize}
    \item The integers under addition: $(\mathbf{Z}, +)$
    \item The rationals under multiplication: $(\mathbf{Q}, *)$
\end{itemize}

\begin{definition}

An abelian group ($G$,\emoji{sparkle}) is a group that also satisfies the commutative property. That is, for all $a, b \in G$, 

\begin{center}
    $a$\emoji{sparkle}$b$ = $b$\emoji{sparkle}$a$
\end{center}

\begin{center}
    \underline{Examples of abelian groups}
\end{center}

\begin{itemize}
    \item $(\mathbf{R}, +)$
    \item $(Z_5, +)$
\end{itemize}


\end{definition}


